%auteur : Damien Frissant

\section{Serveur centralisé versus serveur décentralisé}
\label{sec:centralise}

\subsection{Objectifs}
\label{sec:centralise_Objectifs}
Le but de cette section est de déterminer quelle organisation est mieux adaptés au projet \gls{sae} entre un dispositif centralisé ou décentralisé.

\subsection{Méthodologie de recherche}
\label{sec:centralise_Methodo}
Se référer à la partie \ref{sec:camera_Methodo}.

\subsection{Résultats}
\label{sec:centralise_Resultats}

\subsubsection{Serveur centralisé}
\label{sec:centralise_centralise}
Un système centralisé est composé d'une machine qui centralise et traite toutes les informations.

\enquote{
    Ils sont aussi plus faciles à mettre au point, et à déboguer – quand une erreur arrive, on n’a pas à se demander d’où elle provient.
}{\cite{centraliser}}

Un serveur centralisé à l'avantage de ne pas avoir besoin d'implémenter cette technologie à chaque intersection que l'on souhaite protéger avec \gls{sae}.
Néanmoins, ces systèmes sont critiques et fragile, car un dysfonctionnement a des conséquences sur tous les dispositifs actifs.
La capacité d'un serveur est limité, s'i 'il y a beaucoup de data à traiter il y a des risques de saturation.
De plus il faut une communication ayant un débit correct et une portée suffisante. Cela implique d'éventuelles infrastructures (se référer à la partie \ref{sec:comparaisonProtocoleCommnunication}).
Un serveur centralisé permettrait d'implémenter notre solution au sein d'un poste de commandement sécurité d'une ville. De ce fait,
le dispositif serait moins coûteux à installer pour les collectivités et profiterait à plus de personnes.

\subsubsection{Serveur décentralisé}
\label{sec:centralise_decentralise}
Un système décentralisé peut être vu comme une fourmilière. Par exemple internet est un système décentralisé, où il n'y a des millions de machines autonomes. Il n'y a pas de machine centrale.
Il est facile à mettre en place à des endroits qui ne sont pas relié à un poste de sécurité.
En opposition aux serveurs centralisés, ceux qui sont décentralisés ont une résistance aux attaques, car tous les serveurs sont indépendants.

\subsubsection{Résultats}
\label{sec:centralise_resultat}
En conclusion, il n'y a pas de bon choix entre un système centralisé et décentralisé.
Dans un premier temps, nous allons créer un système décentralisé afin de contrôler le système entièrement. En effet, il est intéressant
d'avoir un prototype fonctionnel dans son ensemble, où l'on peut nous même contrôler la caméra.

Une fois le prototype décentralisé opérationnel, nous pourrons étudier la version centraliser permettant d'équiper des carrefours ayant des caméras à proximité.
