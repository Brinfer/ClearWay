\section{Analyse de la réglementation}
\label{sec:reglementation}

\subsection{Objectifs}
\label{sec:reglementation_Objectifs}

Notre système \gls{sae} a pour but d'être majoritairement implanté dans les espaces publics. 
De ce fait, des réglementations existent légiférant l'utilisation de caméras sur la voie publique.
Cette partie a pour but de documenter la loi en vigueur au 13 novembre 2021 en France, concernant la captation d'image sur les voies publiques et privées. 

\subsection{Méthodologie de recherche}
\label{sec:reglementation_Methodo}

Se référer à la partie \ref{sec:centralise_Methodo}.

\subsection{Résultats}
\label{sec:reglementation_resultats}

L'ajout de matériel de surveillance soulève la question du respect de la vie privée. La \gls{CNIL} est en charge du respect de la vie privée en France.
Par la suite, nous parlerons de vidéoprotection dans le cas d'une caméra se trouvant dans un lieu public 
et de vidéosurveillance pour une caméra placée dans un lieu privé.

\subsubsection{Réglementation sur les caméras dans un environnement privé}
\label{sec:reglementation_privee}

Prenons le cas d'une entreprise. Cette dernière a le droit d'installer des caméras sur les voies de circulation afin de protéger les personnes circulant.
De plus, si l'employeur souhaite filmer des lieux accueillant du public, il doit avoir une autorisation du préfet (s'il fait appelle à une entreprise tierce, aucune autorisation n'est à demander) 
et informer par un affichage les personnes pénétrant dans la zone filmée. Il doit également renseigner la procédure à suivre pour avoir accès aux enregistrements vidéo.
Concernant les données personnelles, l'employeur doit informer les salariés sur les données qu'il collecte. Le temps de conservations des images de vidéosurveillance ne peut excéder un mois.


\subsubsection{Réglementation sur les caméras dans un environnement public}
\label{sec:reglementation_publique}

\paragraph{Avertissements et motifs de la vidéoprotection}
\label{motifs_vidéoprotection}

La vidéoprotection quant à elle est plus restrictive. En effet, il s'agit de filmer en permanence des personnes n'ayant pas forcement conscience qu'elles le sont. 
Il y a l'obligation d'informer toute personne entrant dans une zone protégée et en précisant notamment l’identité et les coordonnées du responsable du traitement et du délégué à la protection des données.
C'est la raison pour laquelle elle doit s'inscrire dans un objectif réglementaire. Différents motifs peuvent justifier l'installation de caméras sur la voie publique, afin de prévenir:
\begin{itemize}
    \item Un acte terroriste
    \item Des atteintes à la sécurité des personnes et des biens dans les lieux exposés à des risques d'agression
    \item Des vols
    \item Le trafic de stupéfiants
    \item etc.
\end{itemize}
Voir notamment les articles L.251-2 et L.223-1 du \gls{CSI} pour plus de détails.

\paragraph{Respect de la vie privée pour la vidéoprotection}
\label{vp_vidéoprotection}

Concernant le respect de la vie privée, les caméras ne doivent pas filmer l'intérieur des immeubles, ni de leurs entrées (article L.251-3 du \gls{CSI}). 
À l'instar de la vidéosurveillance, la durée de conservation des images ne peut excéder un mois, 
mais peut être inférieur selon le motif d'installation du système, ces durées sont détaillées dans l'article L.252-3 du \gls{CSI}.

\paragraph{Qui peut filmer les lieux publics ?}
\label{autorisation_vidéoprotection}
Pour ce qui est des autorisations pour filmer une voie publique, seule une autorité publique est habilitée capter des images. 
En conséquence, les entreprises et les établissements publics peuvent seulement filmer les abords immédiats de leurs bâtiments 
et les lieux susceptibles d’être exposés à des actes de terrorisme comme le précise la \gls{CNIL}.

\subsection{Conclusion}
\label{reglementation_Conclusion}
Après cette recherche concernant l'ajout de caméras de protection ou de surveillance, nous nous rendons compte que notre système est adapté aux deux cas.
En effet, nous n'avons pas besoin d'enregistrer les images. Notre système vidéo se contente de capturer une image, d'analyser l'image. Ensuite une nouvelle image écrase la précédente.
De ce fait, il n'y a pas de problème de conservation des données, car nous ne collectons aucune information. 
\gls{sae} peut être installé aux intersections pour prévenir les accidents des cyclistes et ainsi protéger une partie de la population. 
Au jour où est écrite cette conclusion, des démarches ont été entreprises avec la mairie d'Angers afin de faire des tests en conditions réelles.