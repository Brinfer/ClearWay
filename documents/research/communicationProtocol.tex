\section{Comparaison des protocoles de communication}
\label{sec:comparaisonProtocoleCommnunication}

\subsection{Objectifs}
\label{sec:comparaisonProtocoleCommnunicationObjectifs}

L'objectif de cette étude est de comparer les différents protocoles et logiciels de communication
pouvant être utilisé dans ce projet afin de répondre au mieux aux différents besoins.\newline

Les différents critères étudiés sont :

\begin{itemize}
    \item Le mode de transmission et sa portée
    \item Le débit de la transmission
    \item La consommation énergétique
\end{itemize}

Les différents modes de communication et protocoles étudiés seront :

\begin{itemize}
    \item \nameref{sec:communicationRadio} :
          \begin{itemize}
              \item \Gls{wifi}
              \item \Gls{wimax}
              \item \Gls{bluetooth}
              \item \Gls{lorawan}
          \end{itemize}
    \item \nameref{sec:communicationFilaire} :
          \begin{itemize}
              \item \Gls{cpl}
              \item \Gls{ethernet}
          \end{itemize}
\end{itemize}

\subsection{Résultats}
\label{sec:comparaisonProtocoleCommnunicationResultats}

\subsubsection{Les communications radio}
\label{sec:communicationRadio}

\paragraph{\glsentryname{wifi}}
\label{sec:wifi}

Il existe différentes versions et normes de \gls{wifi}, chacune ayant un débit et une portée différente.
La plupart des cartes réseaux sont adaptées pour la norme \textbf{802.11}, se divisant en différents standards
\cite{wifi}. Il est possible d’utiliser n’importe quel protocole de transport basé sur \gls{ip}.\newline

\begin{table}[H]
    \centering
    \rowcolors{2}{tableColor}{white}
    \begin{tabular}{|c|c|c|c|}
        % Header
        \hline
        \rowcolor{tableColorDark} Normes & Débit de données & Portées & Fréquence d'émission   \\
        \hline

        % Data
        \gls{wifi} 802.11a               & 30 Mbit/s        & 10 m    & 5 GHz                  \\\hline
        \gls{wifi} 802.11b               & 6 Mbit/s         & 300 m   & 2.4 GHz                \\\hline
        \gls{wifi} 802.11g               & 26 Mbit/s        & 70 m    & 2.4 GHz                \\\hline
        \gls{wifi} 802.11n               & 600 Mbit/s       & 70 m    & 2.4 GHz ou 5 GHz       \\\hline
        \gls{wifi} 802.11ac              & 7 Gbit/s         & 30 m    & 2.4 GHz et 5 GHz       \\\hline
        \gls{wifi} 802.11ax              & 10 Gbit/s        & 700 m   & Entre 2.4 GHz et 6 GHz \\\hline
    \end{tabular}
    \label{tab:debitPorteeWifi}
    \caption{Débit et portée du \glsentryname{wifi}}
    \nocite{debitPortee}
\end{table}

\paragraph{\glsentryname{wimax}}
\label{sec:wimax}

Le \gls{wimax} est un ensemble de normes techniques basées sur le standard de transmission radio \textbf{802.16}
permettant la transmission de données \gls{ip}.\newline
Il a pour principale particularité de supporter un haut débit de données (jusqu’à 75 Mbit/s en théorie, mais
excède rarement 20 Mbit/s sur quelques dizaines de kilomètres en condition réelle) sur des distances très importantes,
comprises entre 10 et 50 kilomètres selon les obstacles rencontrés par les ondes, sa capacité à prioriser les usages de
la bande passante disponible entre les différents utilisateurs. \enquote{Ces qualités en font donc une sorte
    de \gls{wifi} survitaminé}{\cite{wimax}}.

\begin{table}[H]
    \centering
    \rowcolors{2}{tableColor}{white}
    \begin{tabular}{|c|c|c|c|}
        % Header
        \hline
        \rowcolor{tableColorDark} Normes & Débit de données & Portées & Fréquence d'émission   \\
        \hline

        % Data
        \gls{wimax} 802.16d              & 12 Mbit/s        & 20 km   & Entre 2 GHz et 11 GHz  \\\hline
        \gls{wimax} 802.16e              & 30 Mbit/s        & 3 km    & Entre 3.5 GHz et 6 GHz \\\hline
    \end{tabular}
    \label{tab:debitPorteeWimax}
    \caption{Débit et portée du \glsentryname{wimax}}
    \nocite{debitPortee}
\end{table}


\paragraph{\glsentryname{bluetooth}}
\label{sec:bluetooth}

Le \gls{bluetooth} sert à la transmission entre divers appareils numériques, avec ou sans connexion,
de données. Contrairement à d'autres technologies de transfert de données, il est
spécialisé dans pour les transferts sur des distances relativement courtes et une basse
consommation. Toutefois, il ne permet pas d'atteindre des débits de transfert élevé.\newline
Le \gls{bluetooth} se distingue du \gls{wifi} par une faible consommation énergétique. Il
ne nécessite en moyenne que 3\% de l'énergie nécessaire au \gls{wifi} pour la même tâche \cite{bluetoothConsumption}.
Le \gls{ble}, qui est une variante, permet une consommation encore plus faible.\newline
On peut résumer les performances du \gls{bluetooth} par ce tableau \cite{debitPortee, ble}.

\begin{table}[H]
    \centering
    \rowcolors{2}{tableColor}{white}
    \begin{tabular}{|c|c|c|c|}
        % Header
        \hline
        \rowcolor{tableColorDark} Normes & Débit de données & Portées & Fréquence d'émission \\
        \hline

        % Data
        \gls{bluetooth} 1.2              & 720 Kbit/s       & 10 m    & 2,4 GHz              \\\hline
        \gls{bluetooth} 2.0              & 2,1 Mbit/s       & 10 m    & 2,4 GHz              \\\hline
        \gls{bluetooth} 3.0              & 2,1 Mbit/s       & 10 m    & 2,4 GHz              \\\hline
        \gls{bluetooth} 4.x              & 3 Mbit/s         & 60 m    & 2,4 GHz              \\\hline
        \gls{bluetooth} 5.0              & 2 Mbit/s         & 200 m   & 2,4 GHz              \\\hline
        \gls{ble}                        & 1 Mbit/s         & 10 m    & 2,4 GHz ou 5 GHz     \\\hline
    \end{tabular}
    \label{tab:debitPorteeBluetooth}
    \caption{Débit et portée du \glsentryname{bluetooth}}
    \nocite{ble}\nocite{debitPortee}
\end{table}

\paragraph{\glsentryname{lorawan}}
\label{sec:lorawan}

Le protocole \gls{lorawan} se veut simple, peu coûteux à implémenter et économe en énergie plutôt que permettre des débits élevés en
utilisant la technologie \gls{lora} \cite{lorawan}. Cette technologie a pour priorité la faible consommation d'énergie.\newline

Tout comme le \gls{wimax}, \gls{lora} qui ont la capacités de prioriser les usages de la bande passante disponible du
réseau des nœuds et augmenter la durée de vie de la batterie.

\begin{table}[H]
    \centering
    \rowcolors{2}{tableColor}{white}
    \begin{tabular}{|c|c|c|c|}
        % Header
        \hline
        \rowcolor{tableColorDark} Normes & Débit de données         & Portées                        & Fréquence d'émission \\
        \hline

        % Data
        \gls{lora}                       & 10.3 kbit/s à 100 kbit/s & de quelques km à plus de 10 km & 868 MHz en Europe    \\\hline
    \end{tabular}
    \label{tab:debitPorteeLora}
    \caption{Débit et portée du \glsentryname{lora}}
    \nocite{debitPortee}
\end{table}

\subsubsection{Les communications filaires}
\label{sec:communicationFilaire}

\paragraph{\glsentryname{cpl}}
\label{sec:cpl}

\blockquote{Le principe des \gls{cpl} consiste à superposer au courant électrique alternatif de 50 ou 60 Hz
    un signal à plus haute fréquence et de faible énergie. Ce deuxième signal se propage sur l’installation
    électrique et peut être reçu et décodé à distance. Ainsi le signal \gls{cpl} est reçu par tout récepteur
    \gls{cpl} de même catégorie se trouvant sur le même réseau électrique. Cette façon de faire comporte cependant un
    inconvénient : le réseau électrique n'est pas adapté au transport de hautes fréquences, car il n'est pas blindé.
    En conséquence, la plus grande partie de l'énergie injectée par le modem \gls{cpl} est rayonnée sous forme
    d'onde radio.}{\cite{cpl}}

La technologie \gls{cpl} permet donc de mettre en place un réseau de communication par-dessus un réseau électrique sans
avoir à mettre en place plus d'infrastructures. Le \gls{cpl} est déjà utilisé dans l'industrie, notamment par \textit{Enedis}
avec les compteurs \textit{Linky}. Les débits atteints sont compris entre 14 et 1 200 Mbit/s.

\begin{table}[H]
    \centering
    \rowcolors{2}{tableColor}{white}
    \begin{tabular}{|c|c|}
        % Header
        \hline
        \rowcolor{tableColorDark} Normes & Débit de données         \\
        \hline

        % Data
        \gls{cpl}                        & 14 Mbit/s à 1 200 Mbit/s \\\hline
    \end{tabular}
    \label{tab:debitCPL}
    \caption{Débit du \glsentryname{cpl}}
    \nocite{cpl}
\end{table}

\paragraph{\glsentryname{ethernet}}
\label{sec:ethernet}

\Gls{ethernet} est le réseau le plus commun, les cartes réseaux les plus courantes le supporte. C'est une technologie
orientée pour des réseaux locaux, généralement dans un même bâtiment. Il existe différent standards avec des débits et des
types de câbles différents \cite{debitEthernet}.

\begin{table}[H]
    \centering
    \rowcolors{2}{tableColor}{white}
    \begin{tabular}{|c|c|c|c|}
        % Header
        \hline
        \rowcolor{tableColorDark} Standard \glsentryname{ethernet} & Débit de données & Câble                                & Distance maximale \\
        \hline

        % Data
        10Base2                                                    & 10 Mbit/s        & Câble coaxial de faible diamètre     & 185 m             \\\hline
        10Base5                                                    & 10 Mbit/s        & Câble coaxial de gros diamètre       & 500 m             \\\hline
        10Base-T                                                   & 10 Mbit/s        & Paire torsadée (catégorie 3)         & 100 m             \\\hline
        100Base-TX                                                 & 100 Mbit/s       & Double paire torsadée (catégorie 5)  & 100 m             \\\hline
        100Base-FX                                                 & 100 Mbit/s       & Fibre optique multimode              & 2 km              \\\hline
        1000Base-T                                                 & 1000 Mbit/s      & Double paire torsadée (catégorie 5e) & 100 m             \\\hline
        1000Base-LX                                                & 1000 Mbit/s      & Fibre optique monomode / multimode   & 500 m - 1 km      \\\hline
        1000Base-SX                                                & 1000 Mbit/s      & Fibre optique multimode              & 500 m             \\\hline
        10GBase-SR                                                 & 10 Gbit/s        & Fibre optique multimode              & 500 m             \\\hline
        10GBase-LX4                                                & 10 Gbit/s        & Fibre optique multimode              & 500 m             \\\hline
    \end{tabular}
    \label{tab:debitEthernet}
    \caption{Débit de l'\glsentryname{ethernet}}
    \nocite{debitEthernet}
\end{table}

\subsection{Conclusion}
\label{sec:comparaisonProtocoleCommnunicationConclusion}

Cette recherche sur les différents moyens de transmettre les informations au sein de \gls{sae} avait pour ambition de trouver le meilleur compromis entre
la consommation énergétique, le débit et la portée.
Il a fallu dans un premier temps réunir les informations, les analyser et les traiter afin de pouvoir les exploiter.\newline

En fonction de comment \gls{sae} sera conçu, c'est-à-dire de manière décentralisée ou centralisée (voir partie \ref{sec:centralise}), alors les
besoins ne seront pas les mêmes.
Il faut toutefois noter la nécessité de faire un choix entre un mode de transmission radio ou filaire.\newline

Dans le cas d'une transmission radio, le \gls{wimax} semble être la technologie la plus adaptée pour un système centralisé, car elle permet d'avoir
une grande portée pouvant couvrir une bonne partie d'une ville et un débit suffisant.
En effet, le \gls{wifi} et le \gls{bluetooth} ont une portée insuffisante, tandis que le \gls{lora} possède un débit trop faible,
ne répondant pas à nos besoins. Toutefois, dans un contexte décentralisé, le \gls{wifi} semble être lui le plus adapté.\newline

Du côté des transmissions filaire, le \gls{cpl} semble le plus adapté, en effet le débit peut être suffisant et sa porté quasiment illimitée, malgré
d'importante perte énergétique. \Gls{ethernet} est plus adapté dans un contexte décentralisé, avec un débit important et une perte énergétique
plus faible que le \gls{cpl}.
