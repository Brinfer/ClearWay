\section{Comparaison des protocoles de communication}
\label{sec:comparaisonProtocoleCommnunication}

\subsection{Objectifs}
\label{sec:comparaisonProtocoleCommnunicationObjectifs}

L'objectif de cette étude est de comparer les différents protocoles et logiciel de communication
pouvant être utilisé dans ce projet afin de répondre au mieux aux différents besoins.\newline

Les différents critères étudiés sont :

\begin{itemize}
    \item Le mode de transmission et sa portée
    \item Le débit de la transmission
    \item La consommation énergétique
\end{itemize}

Les différents modes de communication et protocole étudiés seront :

\begin{itemize}
    % \item Les protocoles internet :
    %       \begin{itemize}
    %           \item \gls{TCP}
    %           \item \gls{SCTP}
    %           \item \gls{UDP}
    %           \item \gls{SSH}
    %           \item \gls{FTP}
    %       \end{itemize}
    \item Les communications radio :
          \begin{itemize}
              \item \gls{wifi}
              \item \gls{bluetooth}
              \item \gls{lora}
          \end{itemize}
\end{itemize}

\subsection{Méthode de recherche}
\label{sec:comparaisonProtocoleCommnunicationMethode}


\subsection{Résultat}
\label{sec:comparaisonProtocoleCommnunicationResultats}

\subsubsection{Les communications radio}
\label{sec:communicationRadio}

\paragraph{\glsentryname{wifi}}
\label{sec:wifi}

\paragraph{\glsentryname{bluetooth}}
\label{sec:bluetooth}

Le \gls{bluetooth} sert à la transmission entre divers appareils numérique, avec ou sans connexion,
de données. Contrairement à d'autre technologie de transfert de données, le \gls{bluetooth} est
spécialisé dans pour les transferts sur des distances relativement courtes et une basse
consommation. Toutefois, il ne permet pas d'atteindre des débits de transfert élevé.

\begin{table}[ht]
    \centering
    \rowcolors{2}{tableColor}{white}
    \begin{tabular}{|c|c|c|c|}
        % Header
        \hline
        \rowcolor{tableColorDark} Normes & Débit de données & Portées & Fréquence d'émission \\
        \hline

        % Data
        \gls{bluetooth} 1.2              & 720 Kbit/s       & 10 m    & 2,4 GHz              \\\hline
        \gls{bluetooth} 2.0              & 2,1 Mbit/s       & 10 m    & 2,4 GHz              \\\hline
        \gls{bluetooth} 3.0              & 2,1 Mbit/s       & 10 m    & 2,4 GHz              \\\hline
        \gls{bluetooth} 4.x              & 3 Mbit/s         & 60 m    & 2,4 GHz              \\\hline
        \gls{bluetooth} 5.0              & 2 Mbit/s         & 200 m   & 2,4 GHz              \\\hline
        \gls{ble}                        & 1 Mbit/s         & 10 m    & 2,4 GHz              \\\hline
    \end{tabular}
    \label{tab:debitPorteeBluetooth}
    \caption{Débit et portée du \glsentryname{bluetooth}}
    \nocite{debitPortee}
    \nocite{ble}
\end{table}

\paragraph{\glsentryname{lora}}
\label{sec:lora}

% \subsubsection{\glsentryname{TCP}}
% \label{sec:tcp}

% \Gls{TCP} permet la transmission bilatérale d’informations.

% \subsubsection{\glsentryname{SCTP}}
% \label{sec:sctp}

% \subsubsection{\glsentryname{UDP}}
% \label{sec:udp}

% \subsubsection{\glsentryname{SSH}}
% \label{sec:ssh}

% \subsubsection{\glsentryname{FTP}}
% \label{sec:ftp}

\subsection{Conclusion}
\label{sec:comparaisonProtocoleCommnunicationConclusion}
