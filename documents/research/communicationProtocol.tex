\section{Comparaison des protocoles de communication}
\label{sec:comparaisonProtocoleCommnunication}

\subsection{Objectifs}
\label{sec:comparaisonProtocoleCommnunicationObjectifs}

L'objectif de cette étude est de comparer les différents protocoles et logiciel de communication
pouvant être utilisé dans ce projet afin de répondre au mieux aux différents besoins.\newline

Les différents critères étudiés sont :

\begin{itemize}
    \item Le mode de transmission et sa portée
    \item Le débit de la transmission
    \item La consommation énergétique
\end{itemize}

Les différents modes de communication et protocole étudiés seront :

\begin{itemize}
    % \item Les protocoles internet :
    %       \begin{itemize}
    %           \item \gls{TCP}
    %           \item \gls{SCTP}
    %           \item \gls{UDP}
    %           \item \gls{SSH}
    %           \item \gls{FTP}
    %       \end{itemize}
    \item Les communications radio :
          \begin{itemize}
              \item \gls{wifi}
              \item \gls{bluetooth}
              \item \gls{lora}
          \end{itemize}
\end{itemize}

\subsection{Méthode de recherche}
\label{sec:comparaisonProtocoleCommnunicationMethode}


\subsection{Résultat}
\label{sec:comparaisonProtocoleCommnunicationResultats}

\subsubsection{Les communications radio}
\label{sec:communicationRadio}

\paragraph{\glsentryname{wifi}}
\label{sec:wifi}

Il existe différente version et norme de \gls{wifi}, chacune ayant un débit et une portée différente.
La plupart des cartes réseaux sont adapté pour la norme \textbf{802.11}, se divisant en différents standards
\cite{wifi}. Il est possible d’utiliser n’importe quel protocole de transport basé sur \gls{ip}.\newline

\begin{table}[ht!]
    \centering
    \rowcolors{2}{tableColor}{white}
    \begin{tabular}{|c|c|c|c|}
        % Header
        \hline
        \rowcolor{tableColorDark} Normes & Débit de données & Portées & Fréquence d'émission   \\
        \hline

        % Data
        \gls{wifi} 802.11a               & 30 Mbit/s        & 10 m    & 5 GHz                  \\\hline
        \gls{wifi} 802.11b               & 6 Mbit/s         & 300 m   & 2.4 GHz                \\\hline
        \gls{wifi} 802.11g               & 26 Mbit/s        & 70 m    & 2.4 GHz                \\\hline
        \gls{wifi} 802.11n               & 600 Mbit/s       & 70 m    & 2.4 GHz ou 5 GHz       \\\hline
        \gls{wifi} 802.11ac              & 7 Gbit/s         & 30 m    & 2.4 GHz et 5 GHz       \\\hline
        \gls{wifi} 802.11ax              & 10 Gbit/s        & 700 m   & Entre 2.4 GHz et 6 GHz \\\hline
        \gls{wifi} 802.16d               & 12 Mbit/s        & 20 km   & Entre 2 GHz et 11 GHz  \\\hline
        \gls{wifi} 802.16e               & 30 Mbit/s        & 3 km    & Entre 3.5 GHz et 6 GHz \\\hline
    \end{tabular}
    \label{tab:debitPorteeWifi}
    \caption{Débit et portée du \glsentryname{wifi}}
    \nocite{debitPortee}
\end{table}

\paragraph{\glsentryname{bluetooth}}
\label{sec:bluetooth}

Le \gls{bluetooth} sert à la transmission entre divers appareils numérique, avec ou sans connexion,
de données. Contrairement à d'autre technologie de transfert de données, il est
spécialisé dans pour les transferts sur des distances relativement courtes et une basse
consommation. Toutefois, il ne permet pas d'atteindre des débits de transfert élevé.\newline
Le \gls{bluetooth} se distingue du \gls{wifi} par une faible consommation énergétique. Il
ne nécessite en moyenne que 3\% de l'énergie nécessaire au \gls{wifi} pour la même tâche \cite{bluetoothConsumption}.
Le \gls{ble}, qui est une variante, permet une consommation encore plus faible.\newline
On peut résumer les performances du \gls{bluetooth} par ce tableau \cite{debitPortee, ble}.

\begin{table}[ht!]
    \centering
    \rowcolors{2}{tableColor}{white}
    \begin{tabular}{|c|c|c|c|}
        % Header
        \hline
        \rowcolor{tableColorDark} Normes & Débit de données & Portées & Fréquence d'émission \\
        \hline

        % Data
        \gls{bluetooth} 1.2              & 720 Kbit/s       & 10 m    & 2,4 GHz              \\\hline
        \gls{bluetooth} 2.0              & 2,1 Mbit/s       & 10 m    & 2,4 GHz              \\\hline
        \gls{bluetooth} 3.0              & 2,1 Mbit/s       & 10 m    & 2,4 GHz              \\\hline
        \gls{bluetooth} 4.x              & 3 Mbit/s         & 60 m    & 2,4 GHz              \\\hline
        \gls{bluetooth} 5.0              & 2 Mbit/s         & 200 m   & 2,4 GHz              \\\hline
        \gls{ble}                        & 1 Mbit/s         & 10 m    & 2,4 GHz ou 5 GHz     \\\hline
    \end{tabular}
    \label{tab:debitPorteeBluetooth}
    \caption{Débit et portée du \glsentryname{bluetooth}}
    \nocite{ble}\nocite{debitPortee}
\end{table}

\paragraph{\glsentryname{lora}}
\label{sec:lora}

% \subsubsection{\glsentryname{TCP}}
% \label{sec:tcp}

% \Gls{TCP} permet la transmission bilatérale d’informations.

% \subsubsection{\glsentryname{SCTP}}
% \label{sec:sctp}

% \subsubsection{\glsentryname{UDP}}
% \label{sec:udp}

% \subsubsection{\glsentryname{SSH}}
% \label{sec:ssh}

% \subsubsection{\glsentryname{FTP}}
% \label{sec:ftp}

\subsection{Conclusion}
\label{sec:comparaisonProtocoleCommnunicationConclusion}
