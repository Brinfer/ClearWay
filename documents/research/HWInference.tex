\section{Temps de traitement par Intelligence Artificielle}
\label{sec:hwInference}

\subsection{Objectifs}
\label{sec:hwInference_Objectifs}

\textit{Cette partie est écrite à la suite des explorations techniques de l'analyse d'images pour \gls{IA}}

Le traitement d'image par \gls{IA} nécessite beaucoup de ressources de calcul. 
Appliqué à un système embarqué ayant pour but de diminuer les accidents entre les cyclistes et les automobilistes aux intersections, 
une latence minimale est primordiale.
Cette dernière peut être réduite en ayant un meilleur micro contrôleur, avec plus de \gls{RAM}, un processeur plus puissance ou une carte graphique. 
Néanmoins, le monde de l'embarqué est très restrictif et nous souhaitons améliorer les performances de traitement sans changer le matériel qui est à notre disposition.
De ce fait, nous ne pouvons diminuer la latence en agissant sur les algorithmes utilisés et sur les périphériques de notre système.

\subsection{Méthodologie de recherche}
\label{sec:hwInference_Methodo}

Se référer à la partie \ref{sec:centralise_Methodo}.

\subsection{Résultats}
\label{sec:hwInference_resultats}

\subsubsection{Comparatif des algorithmes}
\label{sec:hwInference_comp_algo}

À la suite des explorations techniques, il s'est avéré que la reconnaissance d'image est relativement lente. Dans la partie \ref{sec:comparaisonIA} nous utilisons l'algorithme YOLO v3.
Nous avons consacré notre étude sur trois algorithmes : YOLO v3, Tiny YOLO v3 et Tiny YOLO v2. Ci-dessous, le tableau comparatif en implémentant les algorithmes avec la capture d'images sur deux minutes :

\begin{table}[ht!]
    \centering
    \begin{tabular}{|l|l|l|}
    \hline
                 & IPS  & Précision \\ \hline
    YOLO v3      & 0,13 & 90\%      \\ \hline
    Tiny YOLO v3 & 1,49 & 56\%      \\ \hline
    Tiny Yolo v2 & 1,49 & 76\%      \\ \hline
    \end{tabular}
    \caption{Comparatif des différents algorithmes avec capture caméra}
    \label{fig:comparatifAlgoAvecCam}
    \end{table}

Les résultats ci-dessus nous permettent d'éliminer l'utilisation de YOLO v3 pour le Raspberry Pi 4. 
En effet, malgré la haute précision de détection, on obtenait 0,13 Image Par Seconde (IPS). 
Cela n'est pas suffisant, pour rappel, la partie \ref{sec:camera_resultats} recommande 2 IPS.
En utilisant Tiny YOLO, nous atteignons 1.49 IPS, ce qui est nécessaire pour notre projet, mais pas suffisant pour la sécurité des usagés.
Pour la suite du projet, nous allons utiliser l'algorithme Tiny YOLO. 
En nous appuyant sur les données du tableau, nous utiliserons la version 2 de cet algorithme qui a une meilleure fiabilité de détection que la version 3.

\subsubsection{Utilisation d'un hardware d'\glsentryname{inference}}
\label{sec:hwInference_hwinf}
L'utilisation d'un hardware d'\gls{inference} permettrait de gagner en performance en réduisant le temps d'\gls{inference}.
Pour notre projet, nous avons besoin de diminuer l'\gls{inference} pour satisfaire le cahier des charges du projet.

Les hardwares d'\gls{inference} sont conçus pour accélérer les applications d'\gls{IA}.
Il existe différents types d'hardware d'\gls{inference}: \gls{GPU}, \gls{FPGA} et \gls{VPU}. 
Chacune de ces technologies présente des avantages et des limites.

\paragraph{\glsentryname{GPU}}
\label{sec:GPU_hwInf}
Le \gls{GPU} a une architecture parallèle adaptée pour accélérer l'\gls{inference}. 
Il a une consommation énergétique importante, ce qui n'est pas souhaitable pour un système embarqué. 
De plus, cette technologie est coûteuse d'un point de vue matériel.

\paragraph{\glsentryname{FPGA}}
\label{sec:FPGA_hwInf}

Le \gls{FPGA} est largement déployé dans les applications de vision industrielle et constitue la base de nombreuses caméras de vision industrielle et cartes d'acquisition d'images. 
C'est un bon compromis entre la flexibilité et la programmabilité d'un logiciel exécuté sur un processeur à usage général et la vitesse et l'efficacité d'une application conçue sur mesure. 
Cette technologie est très coûteuse d'un point de vue des compétences et matériel, ce qui est dommageable pour notre projet.
Une limitation à l'utilisation du \gls{FPGA} dans la vision industrielle est que la programmation \gls{FPGA}, cette dernière nécessite une compétence hautement spécialisée.

\paragraph{\glsentryname{VPU}}
\label{sec:VPU_hwInf}

Une technologie plus récente, le \gls{VPU} est un type de système sur puce conçu pour l'acquisition et l'interprétation d'informations visuelles. 
Le \gls{VPU} cible les applications mobiles et est optimisé pour une petite taille et une efficacité énergétique avantageuse. 
Cette faible consommation d'énergie est propice aux applications embarquées. 
Le processeur \gls{VPU} Movidius Myriad d'Intel \cite{Movidius} par exemple, peut s'interfacer avec un capteur d'image, 
pré-traiter les données d'image capturées, transmettre les images résultantes via un réseau de neurones pré-entraîné et produire un résultat. 
Cela est possible grâce à la combinaison des cœurs de processeur traditionnels et des cœurs de traitement vectoriels pour accélérer la logique hautement ramifiée typique des réseaux de neurones profonds.

Ces accélérateurs fonctionnent en assistant l'unité de traitement informatique du périphérique en prenant en charge la charge mathématique nécessaire à l'exécution de modèles d'apprentissage en profondeur. 
La dernière génération de \gls{VPU} d'Intel comprend 16 cœurs de traitement (appelés cœurs SHAVE) et un accélérateur matériel de réseau neuronal profond dédié pour les applications de vision et d'\gls{inference} d'\gls{IA} hautes performances.
De plus, le \gls{VPU} consomme peu d'énergie et est portable, ces points sont des avantages pour les systèmes embarqués.

\subsection{Conclusion}
\label{hwInference_Conclusion}
Au vu des résultats obtenus lors de nos tests, il faudrait utiliser l'algorithme Tiny YOLO v2. En cumulant la puissance du Raspberry Pi 4 ayant 4 Go de \gls{RAM} avec cet algorithme, 
nous obtenons 1,49 IPS, ce qui est acceptable, mais pas suffisant pour garantir la fiabilité du système.
Coupler les technologies précédentes avec un hardware d'\gls{inference} de type \gls{VPU} pourrait permettre de diminuer drastiquement le temps d'\gls{inference}.
L'Intel® Neural Compute Stick 2 \cite{Movidius} ayant le processeur Movidius™ Myriad™ X pourrait être un choix judicieux, car largement documenté, puissant et abordable.