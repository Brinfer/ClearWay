\newglossaryentry{TCP}
{
    name={Transmission Control Protocol},
    description={Protocole au sein de la suite de protocoles Internet, un ensemble de normes
            permettant aux systèmes de communiquer sur Internet. Il est classé en tant que protocole de
            "couche de transport" car il crée et maintient des connexions entre hôtes.},
    plural={TCP},
    text={TCP},
    firstplural={Transmission Control Protocol (TCP)},
    first={Transmission Control Protocol (TCP)},
}

\newglossaryentry{UDP}
{
    name={User Datagram Protocol},
    description={Protocoles de télécommunication utilisés par Internet. Il fait partie de la couche
            transport du modèle OSI, quatrième couche de ce modèle, comme \glsentrytext{TCP}.},
    plural={UDP},
    text={UDP},
    firstplural={User Datagram Protocol (UDP)},
    first={User Datagram Protocol (UDP)},
}

\newglossaryentry{SSH}
{
    name={Secure Shell},
    description={Protocole de communication et programme informatique, permettant la connexion d'une
            machine distante (serveur) via une liaison sécurisée dans le but de transférer des fichiers
            ou des commandes en toute sécurité.},
    plural={SSH},
    text={SSH},
    firstplural={Secure Shell (SSH)},
    first={Secure Shell (SSH)},
}

\newglossaryentry{FTP}
{
    name={File Transfer Protocol},
    description={Logiciel utilisé dans le transfert de fichiers entre deux ordinateurs. Il est, avec le
            client FTP, l'une des deux composantes d'un transfert de fichiers via le langage FTP.},
    plural={FTP},
    text={FTP},
    firstplural={File Transfer Protocol (FTP)},
    first={File Transfer Protocol (FTP)},
}

\newglossaryentry{SCTP}
{
    name={Stream Control Transmission Protocol},
    description={Protocole de transport, SCTP est équivalent dans un certain sens au \glsentrytext{TCP} ou à
            l'\gls{UDP}. En effet, il fournit des services similaires à \glsentrytext{TCP}, assurant
            optionnellement la fiabilité, la remise en ordre des séquences et le contrôle de congestion.},
    plural={SCTP},
    text={SCTP},
    firstplural={Stream Control Transmission Protocol (SCTP)},
    first={Stream Control Transmission Protocol (SCTP)},
}

\newglossaryentry{CROP}{
    name={Coefficient multiplicateur},
    description={Fait référence à la taille du capteur photographique. Plus le capteur est petit, plus le coefficient est élevé.
            Cela veut dire qu'a longueur focale égale, un capteur plein format (coefficient égal à 1x)
            aura un champs de vision plus grand qu'un capteur APS d'un boitier Nikon (coefficient de 1.95x).},
    plural={coefficient multiplicateur},
    text={coefficient multiplicateur},
}

\newglossaryentry{wifi}{
    name={Wi-Fi},
    description={Ensemble de protocoles de communication sans fil},
}

\newglossaryentry{bluetooth}{
    name={Blueetooth},
    description={Norme de télécommunications permettant l'échange bidirectionnel de données à courte distance en
            utilisant des ondes radio.},
}

\newglossaryentry{ble}{
    name={Bluetooth Low Energy},
    description={Le Bluetooth Low Energy est une alternative au \glsentrytext{bluetooth} classique, avec une
            réduction du coût et de la consommation en puissance, tout en conservant une portée de
            communication équivalente.},
    plural={BLE},
    text={BLE},
    firstplural={Bluetooth Low Energy (BLE)},
    first={Bluetooth Low Energy (BLE)},
}

\newglossaryentry{lora}{
    name={Long Range},
    description={Protocole de télécommunication radio permettant la communication à bas débit d'objets connectés.
            Le signal radio est émis sur une grande largeur spectrale, pour limiter au maximum le risque d'interférence
            avec des signaux parasites.},
    plural={LoRa},
    text={LoRa},
    firstplural={Long Range (LoRa)},
    first={Long Range (LoRa)},
}

\newglossaryentry{ip}{
    name={Internet Protocol},
    description={Famille de protocoles de communication de réseaux informatiques conçus pour être utilisés sur Internet.},
    plural={IP},
    text={IP},
    firstplural={Internet Protocol (IP)},
    first={Internet Protocol (IP)},
}

\newglossaryentry{wimax}{
    name={Worldwide Interoperability for Microwave Access},
    description={Standard de communication sans fil. Aujourd'hui il est surtout utilisé comme système de transmission et
            d'accès à Internet à haut débit, portant sur une zone géographique étendue.},
    plural={WIMAX},
    text={WIMAX},
    firstplural={Worldwide Interoperability for Microwave Access (WIMAX)},
    first={Worldwide Interoperability for Microwave Access (WIMAX)},
}

\newglossaryentry{lorawan}{
    name={Long Range Wide-Area Network},
    description={Protocole de télécommunication permettant la communication à bas débit, par radio, d'objets à faible
            consommation électrique communiquant selon la technologie \glsentrytext{lora} et connectés à l'Internet.},
    plural={LoRaWAN},
    text={LoRaWAN},
    firstplural={Long Range Wide-Area Network (LoRaWAN)},
    first={Long Range Wide-Area Network (LoRaWAN)},
}

\newglossaryentry{epf}{
    name={Equivalent Plein Format},
    description={Equivalent d'une longueur focale lorsqu'elle est ramenée à la taille d'un capteur plein format.},
    text={EPF},
    first={Equivalent Plein Format (EPF)},
}

\newglossaryentry{video de reference}{
    name={Vidéo de référence},
    description={Cette vidéo ainsi que ses variantes sont disponibles à l'adresse suivante :
        \href{https://drive.google.com/drive/folders/1zz_7GOPkGNsO2MbojXfIZwNzYIrYXeyR}{Lien Google Drive}},
    text={vidéo de référence},
}

\newglossaryentry{CPU}{
    name={CPU},
    description={(Central Processing Unit) Il s'agit d'un processeur. C'est un composant présent dans de nombreux dispositifs électroniques qui exécute les instructions machine des programmes informatiques.},
    text={CPU},
}

\newglossaryentry{GPU}{
    name={GPU},
    description={(Graphics Processing Unit) Il s'agit d'un processeur dédié au traitement des données graphiques.},
    text={GPU},
}
