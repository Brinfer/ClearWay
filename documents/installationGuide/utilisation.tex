\part{Guide d'utilisation}
La présente section a pour but de décrire l'utilisation du \gls{sae}.

\section{\glsentryname{sae} - Description du service}
\label{sec:description_clearWay}

\gls{sae} est composé de trois éléments :
\begin{itemize}
    \item Le module caméra comprend une caméra (\gls{RPiCamera}), un socle en bois, un servomoteur (\gls{SG90})
          et un dispositif en plastique permettant de faire la liaison entre le servomoteur et la caméra.
    \item Un boîtier noir où se trouve une carte \gls{RPiCard}.
    \item Le panneau en bois intégrant des LEDs permettant les automobilistes qu'un cycliste arrive.
\end{itemize}

\section{Exécution de \glsentryname{sae}}
\label{sec:execution_clearWay}

\gls{sae} est actuellement un prototype.

\subsection{Utilisation du \glsentrytext{ssh}}
\label{sec:utilisationSSH}

Pour une exécution sur la \gls{raspberry} à distance, il faut établir une connexion \gls{ssh}
Pour se faire, le plus simple est d'être connecté au même réseau que cette dernière.

\subsubsection{Trouver l'adresse \glsentrytext{ip}}
\label{sec:trouverIP}

\paragraph{À partie du \textit{hostname}}

Si vous connaissez le \textit{hostname} de votre \gls{raspberry} : \mintinline{bash}{ping <hostname>.local}.\\
L'adresse \gls{ip} s'affichera dans le terminal.

\subparagraph{Exemple}

Pour une \gls{raspberry} dont le \textit{hostname} est \underline{raspberrypi} (valeur par défaut à
l'installation) :

\begin{minted}{text}
    $ ping raspberrypi.local

    PING raspberrypi.local (123.123.123.123) 56(84) octets de données.
    64 octets de par10s33-in-f14.1e100.net (123.123.123.123) : icmp_seq=1 ttl=115 temps=19.3 ms
    64 octets de fra02s18-in-f14.1e100.net (123.123.123.123) : icmp_seq=2 ttl=115 temps=24.3 ms
    ...
\end{minted}

L'\gls{ip} nous est alors donné et est \texttt{123.123.123.123}.

\paragraph{À partie de votre adresse \glsentrytext{ip}}

Sinon retrouver votre adresse \gls{ip}\footnote{Référer vous à la documentation de votre OS}, afin de déterminer
la plage de sous-réseau, le et utiliser la commande \mintinline{bash}{nmap -sn <subnet>/24}.
L'ensemble des appareils sur votre réseau seront "pinger" :

\subparagraph{Exemple}

Si mon \gls{ip} est \texttt{123.123.123.5} alors mon sous-réseau est \texttt{123.123.123.0}.

\begin{minted}{text}
    # sudo nmap -sn 123.123.123.0/24

    Starting Nmap 7.92 ( https://nmap.org ) at 2022-01-07 14:58 EAT
    Nmap scan report for 123.123.123.1
    Host is up (0.0055s latency).
    MAC Address: 50:0F:F5:3E:22:50 (Unknown)
    Nmap scan report for 123.123.123.106
    Host is up (0.010s latency).
    MAC Address: C4:D9:87:BA:88:17 (Intel Corporate)
    Nmap scan report for 123.123.123.105
    Host is up.
    MAC Address: B8:37:EB:EA: E0:D5 (Raspberry Pi Foundation)
    Nmap scan report for 123.123.123.200
    Host is up (0.0049s latency).
    Nmap done: 256 IP addresses (4 hosts up) scanned in 205.26 seconds.
\end{minted}

Dans notre exemple ci-dessus, la commande \texttt{nmap} a trouvé un périphérique \gls{raspberry} :

\begin{minted}{text}
    MAC Address: B8:37:EB:EA: E0:D5 (Raspberry Pi Foundation)
    Nmap scan report for 123.123.123.200
\end{minted}

\subsubsection{Connexion}

Pour se connecter en \gls{ssh} à la \gls{raspberry}, il faut connaitre l'utilisateur\footnote{Par défaut sur une
    \gls{raspberry}, l'utilisateur est \underline{pi}.} auquel on veut se connecter et l'\gls{ip} de la \gls{raspberry}
(voir section \nameref{sec:trouverIP}).\\
La commande à exécuter est :

\begin{minted}{bash}
    ssh <user>@<ip_address>
\end{minted}

Il vous sera alors demandé de confirmer si vous souhaiter vous connecter, et ensuite, le mot de passe de
l'utilisateur\footnote{Par défaut pour l'utilisateur \underline{pi}, le mot de passe est \underline{raspberry}}.

\paragraph{Copie de fichier de l'ordinateur vers la \glsentryname{raspberry}}
\label{sec:copieVersRaspberry}

Pour copier un fichier depuis votre ordinateur vers la \gls{raspberry} (ou une quelconque cible distante), il faut
utiliser la commande \texttt{scp} qui utilise le \gls{ssh}.\\
La commande est :

\begin{minted}{bash}
    scp <file_to_send> <user>@<ip_address>:<path_on_the_raspberry>
\end{minted}

Pour copier un répertoire, ajouter l'argument \texttt{-r} à la commande \texttt{scp}.

\subsection{Paramètres}

\subsubsection{Paramètres obligatoires}
\label{sec:executionArg_clearWay}

Certains arguments sont nécessaires afin d'utiliser \gls{sae}. Le tableau ci-dessous répertorie ces derniers.

\begin{table}[H]
    \centering
    \rowcolors{2}{tableColor}{white}
    \begin{tabularx}{\linewidth}{|c|c|X|}
        \hline
        \rowcolor{tableColorDark}
        Argument                  & Alias       & \multicolumn{1}{c|}{\cellcolor{tableColorDark}Description}                                                                                                                    \\ \hline
        \texttt{-{}-yolo-weights} &             & Chemin jusqu'au fichier \texttt{.weight} de \gls{yolo}. Requis si la valeur n'est pas fournie dans le fichier \gls{toml} (voir section \nameref{sec:executionTOML_clearWay}). \\ \hline
        \texttt{-{}-yolo-cfg}     &             & Chemin jusqu’au fichier \texttt{.cfg} de \gls{yolo}. Requis si la valeur n'est pas fournie dans le fichier \gls{toml} (voir section \nameref{sec:executionTOML_clearWay}).    \\ \hline
        \texttt{-{}-config}       & \texttt{-c} & Chemin vers le fichier de configuration de \gls{sae}. Requis si au moins un des chemins vers un fichier \gls{yolo} n'est pas fournie.                                         \\ \hline
    \end{tabularx}
    \caption{Arguments obligatoires de \gls{sae}}
\end{table}

\subsubsection{Paramètres optionnels}
\label{sec:executionOption_clearWay}
\gls{sae} est livré avec plusieurs options à renseigner lors du démarrage du programme. La liste ci-dessous répertorie ces dernières.

\begin{table}[H]
    \centering
    \rowcolors{2}{tableColor}{white}
    \begin{tabularx}{\linewidth}{|c|c|X|c|c|c|}
        % Header
        \hline
        \cellcolor{tableColorDark}                           & \cellcolor{tableColorDark}                        & \multicolumn{1}{c|}{\cellcolor{tableColorDark}}                                                                 & \cellcolor{tableColorDark}                                    & \multicolumn{2}{c|}{\cellcolor{tableColorDark}Support}                                          \\ \cline{5-6}
        \multirow{-2}{*}{\cellcolor{tableColorDark}Argument} & \multirow{-2}{*}{\cellcolor{tableColorDark}Alias} & \multicolumn{1}{c|}{\multirow{-2}{*}{\cellcolor{tableColorDark}Description}}                                    & \multirow{-2}{*}{\cellcolor{tableColorDark}Valeur par défaut} & \multicolumn{1}{c|}{\cellcolor{tableColorDark}Ordinateur} & \cellcolor{tableColorDark}Raspberry \\ \hline
        % Data
        \texttt{-{}-camera-angle}                            &                                                   & L'angle de la caméra en degrés.                                                                                 & \texttt{75}                                                   &                                                           & X                                   \\\hline
        \texttt{-{}-help}                                    & \texttt{-h}                                       & Affiche les arguments et options possibles.                                                                     &                                                               & X                                                         & X                                   \\\hline
        \texttt{-{}-input-path}                              & \texttt{-i}                                       & Chemin vers la vidéo à analyser si la caméra n'est pas utilisée.                                                &                                                               & X                                                         & X                                   \\\hline
        \texttt{-{}-no-gpio}                                 &                                                   & Les GPIOs ne seront pas utilisés. Uniquement les logs seront affichés. Incompatible avec \texttt{-{}-use-gpio}. & \texttt{true}                                                 & X                                                         & X                                   \\\hline
        \texttt{-{}-on-raspberry}                            &                                                   & Informe le programme que la \gls{raspberry} sera utilisée.                                                      & \texttt{false}                                                &                                                           & X                                   \\\hline
        \texttt{-{}-output-path}                             & \texttt{-o}                                       & Chemin vers le dossier qui contraindra le résultat de la vidéo analysée                                         &                                                               & X                                                         & X                                   \\\hline
        \texttt{-{}-panel-gpios}                             &                                                   & Les GPIOs à utiliser                                                                                            & \texttt{5}                                                    &                                                           & X                                   \\\hline
        \texttt{-{}-see-rtp}                                 &                                                   & Affiche une fenêtre avec l'analyse en temps réelle.                                                             & \texttt{false}                                                & X                                                         & X                                   \\\hline
        \texttt{-{}-servo-gpio}                              &                                                   & La broche GPIO utilisé par le servomoteur.                                                                      & \texttt{12}                                                   &                                                           & X                                   \\\hline
        \texttt{-{}-use-gpio}                                &                                                   & Les GPIOs seront utilisés, les logs seront également affichés. Incompatible avec \texttt{-{}-no-gpio}.          & \texttt{false}                                                & X                                                         &                                     \\\hline
        \texttt{-{}-verbosity }                              & \texttt{-v}                                       & Le niveau de verbosité, valeur possibles : \texttt{WARNING}, \texttt{INFO} ou \texttt{DEBUG}.                   & \texttt{INFO}                                                 & X                                                         & X                                   \\\hline
        \texttt{-{}-version}                                 & \texttt{-V}                                       & Affiche la version de \gls{sae}                                                                                 &                                                               & X                                                         & X                                   \\\hline
    \end{tabularx}
    \label{tabOptClearway}
    \caption{Arguments optionnels de \gls{sae}}
\end{table}

\paragraph{Exemple}

\begin{minted}{bash}
    clearway \
    --no-gpio \
    --gpios 5 6 \
    --servo-gpio 12 \
    --verbosity DEBUG \
    --output-path output/video1.mp4 \
    --size 320 \
    --yolo-weights resources/yolov2-tiny.weights \
    --yolo-cfg resources/yolov2-tiny.cfg
\end{minted}

\subsection{Utilisation d'un fichier \glsentrytext{toml}}
\label{sec:executionTOML_clearWay}

\gls{sae} peut être configuré à l'aide d'un fichier \gls{toml}. Ce fichier doit être composé
d'une section principale \texttt{clearway}, elle-même composait de plusieurs sous-sections, chacune avec différentes clés.

\begin{table}[H]
    \rowcolors{2}{tableColor}{white}
    \begin{tabularx}{\linewidth}{|l|c|X|}
        \hline
        \multicolumn{1}{|c|}{\cellcolor{tableColorDark}Clée} & \multicolumn{1}{c|}{\cellcolor{tableColorDark}Type} & \multicolumn{1}{c|}{\cellcolor{tableColorDark}Description} \\ \hline
        \multicolumn{3}{|c|}{\cellcolor{tableColorDark}sous-section \texttt{gpio}}                                                                                              \\ \hline
        \texttt{camera\_angle}                               & entier                                              & L'angle de la caméra en degrés                             \\ \hline
        \texttt{servo\_gpio}                                 & entier                                              & le GPIO utilisé par le servomoteur                         \\ \hline
        \texttt{use\_gpio}                                   & booléen                                             & Les GPIOs à utiliser, les logs seront également affichés.  \\ \hline
        \multicolumn{3}{|c|}{\cellcolor{tableColorDark}sous-section \texttt{logging}}                                                                                           \\ \hline
        \texttt{format}                                      & chaine de caractère                                 & Format des messages de log \cite{loggingFormat}            \\ \hline
        \texttt{path}                                        & chaine de caractère                                 & Chemin vers le fichier des logs                            \\ \hline
        \texttt{verbosity}                                   & chaine de caractère                                 & Le niveau de verbosité                                     \\ \hline
        \multicolumn{3}{|c|}{\cellcolor{tableColorDark}sous-section \texttt{ia}}                                                                                                \\ \hline
        \texttt{size}                                        & entier                                              & La taille de l'image (320 ou 416 recommandés)              \\ \hline
        \texttt{see\_rtp}                                    & booléen                                             & Affichage d'une fenêtre avec l'analyse en temps réelle     \\ \hline
        \texttt{yolo\_cfg}                                   & chaine de caractère                                 & Chemin jusqu'au fichier \texttt{.cfg} de \gls{yolo}        \\ \hline
        \texttt{yolo\_weight}                                & chaine de caractère                                 & Chemin jusqu'au fichier \texttt{.weight} de \gls{yolo}     \\ \hline
    \end{tabularx}
    \caption{Clés du fichier \glsentrytext{toml} de \glsentryname{sae}}
\end{table}

Les clés \texttt{yolo\_cfg} et \texttt{yolo\_weight} sont obligatoires pour le fonctionnement sauf si ces valeurs sont
passées en ligne de commande, de la même manière on peut surcharger les valeurs contenues dans le fichier
(voir section \nameref{sec:executionArg_clearWay}).\newline

\subsubsection{Exemple}

\begin{minted}{toml}
    [clearway]
        [clearway.gpio]
            use_gpio = false
            panel_gpios = [5, 6]
            camera_angle = 75
            servo_gpio = 12
        [clearway.ai]
            yolo_cfg = "resources/yolov2-tiny.cfg"
            yolo_weights = "resources/yolov2-tiny.weights"
            on_raspberry = false
            size = 320
            see_rtp = false
        [clearway.log]
            verbosity = "DEBUG"
            format = "%(asctime)s [%(filename)s:%(lineno)d] %(levelname)s >> %(message)s"
            path = "clearway.log"
\end{minted}