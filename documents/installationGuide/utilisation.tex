\part{Guide d'utilisation}
La présente section a pour but de décrire l'utilisation du \gls{sae}.

\section{\glsentryname{sae} - Description du service}
\label{sec:description_clearWay}

\gls{sae} est composé de trois éléments :
\begin{itemize}
    \item Le module caméra comprend une caméra (\gls{RPiCamera}), un socle en bois, un servomoteur (\gls{SG90})  
    et un dispositif en plastique permettant de faire la liaison entre le servomoteur et la caméra.
    \item Un boîtier noir où se trouve une carte \gls{RPiCard}.
    \item Le panneau en bois intégrant des LEDs permettant les automobilistes qu'un cycliste arrive.
\end{itemize}

\section{Exécution de \glsentryname{sae}}
\label{sec:execution_clearWay}

Le service est actuellement un prototype. Pour une exécution sur le \gls{RPiCard} :
Afin de l'exécuter il faut établir une connexion ssh avec le \gls{RPiCard} et être connecté au même réseau que ce dernier.
Trouver l'adresse IP de votre Raspberry :
\begin{itemize}
    \item Connaître l'adresse IP du réseau local, par exemple : 169.25.2.11
    \item Trouver votre Raspberry en tapant cette commande dans un terminal bash, par exemple 169.25.2.168 : 
\begin{minted}{bash}
    nmap -sn 169.25.2.0/24
\end{minted}
    \item Se connecter en ssh à la \gls{RPiCard} :
\begin{minted}{bash}
    ssh pi@169.25.2.168
\end{minted}
\end{itemize}

\subsection{Arguments nécessaires pour l'exécution du programme}
\label{sec:executionArg_clearWay}
Certains arguments sont nécessaires afin d'utiliser \gls{sae}.
