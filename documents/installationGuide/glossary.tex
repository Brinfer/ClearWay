\newglossaryentry{python}
{
    name={Python},
    description={Langage de programmation interprété, interactif et orienté objet. \cite{python}},
    plural={Python},
}

\newglossaryentry{pip}
{
    name={pip},
    description={Installateur de \glspl{paquet} pour \gls{python}. \cite{pip}},
    plural={pip},
    text={pip},
}

\newglossaryentry{paquet}
{
    name={Package},
    description={Contient tous les fichiers dont vous avez besoin pour un module.},
    plural={packages},
    text={package},
}

\newglossaryentry{docstring}
{
    name={Docstring},
    description={Les chaînes de caractères utilisées juste après la définition d'une fonction, d'une méthode, d'une
            classe ou d'un module. Elles sont utilisées pour documenter le code.},
    plural={docstrings},
    text={docstring},
}

\newglossaryentry{mock}
{
    name={Mock},
    description={Terme spécialisé de la programmation informatique qui sert à désigner un objet virtuel, de quelque
            nature que ce soit, qui a toutes les attitudes, toutes les apparences et tous les mouvements de son double
            réel.},
    plural={mocks},
    text={mock},
}

\newglossaryentry{framework}
{
    name={Framework},
    description={Ensemble d'outils et de composants logiciels à la base d'un logiciel ou d'une application.},
    plural={frameworks},
    text={framework},
}
    
\newglossaryentry{SG90}
{
    name={MicroServo SG90},
    description={Le servomoteur sert à orienter la caméra horizontalement.},
    plural={SG 90},
}

\newglossaryentry{RPiCamera}
{
    name={Raspberry Pi v2.1 8 MP 1080p Module Caméra},
    description={Caméra de la marque Raspberry Pi.},
    plural={Raspberry Pi v2.1 8 MP 1080p Module Caméra},
}

\newglossaryentry{RPiCard}
{
    name={Raspberry Pi 4 Model B},
    description={Le microcontrôleur du service},
    plural={Raspberry Pi 4 Model B},
}