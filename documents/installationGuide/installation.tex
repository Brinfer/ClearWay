\part{Guide d'installation}

L'installation de \gls{sae} nécessite l'installation de \gls{python} et de \gls{pip}.\\
La version minimale de \gls{python} demandé est \textbf{3.7.*}, les versions \textbf{3.9} et supérieur n'ont pas été
testées.\\
Référer vous à la documentation de votre distribution (ou OS) pour savoir comment les installer.

\section{Depuis les fichiers sources}

Dans toute cette section, nous considérons que le terminal est placé au niveau du dossier du projet.

\begin{minted}{bash}
    cd path/vers/le/dossier/du/projet
\end{minted}

\subsection{Installation des \glsentryplural{paquet}}

Afin de pouvoir utiliser \gls{sae}, certains \glspl{paquet} sont nécessaires.\newline

L'installation des \glspl{paquet} peut être rapidement faite en utilisant le fichier \verb=requirements.txt= se
trouvant à la racine du projet :

\begin{minted}{bash}
    python3 -m pip install -r requirements.txt
\end{minted}

\subsubsection{Liste des \glsentryplural{paquet}}

\begin{itemize}
    \item Formater et linter
          \begin{description}
              \item[black] Formateur de code \gls{python} \cite{black}
              \item[flake8] Combinaison d'outils pour vérifier la code par rapport au style de codage PEP8
                  \cite{flake8, pep8}
                  \begin{description}
                      \item[flake8-bandit] Tests de sécurité automatisés \cite{flake8_bandit}
                      \item[flake8-import-order] Vérifie l'ordre de vos importations \cite{flake8_import_order}
                      \item[flake8-docstrings] Vérifie la conformité aux conventions des \gls{docstring} \gls{python}
                          \cite{flake8_docstrings}
                      \item[flake8-bugbear] Trouve les probables bogues et problèmes de conception dans le code
                          \cite{flake8_bugbear}
                      \item[flake8-return] Vérifie les valeurs de retour des fonctions dans le code \cite{flake8_return}
                      \item[pep8-naming] Vérifie le code par rapport aux conventions de nommage PEP8 \cite{pep8_naming, pep8}.
                  \end{description}
              \item[mypy]
                  Vérificateur de type statique pour \gls{python}. \cite{mypy}.
          \end{description}

    \item Nécessaire pour le fonctionnement
          \begin{description}
              \item[transitions] Une implémentation légère et orientée objet de la machine à états en \gls{python} \cite{transition}
              \item[RPi.GPIO] Fournit une classe pour contrôler le GPIO sur une Raspberry Pi \cite{rpi_gpio}.
                  {\color{red}Il ne sera pas installé avec le fichier \verb=requirements.txt=, car non disponible pour
                  les architectures autres que ARM}
              \item[imutils] Une série de fonctions de commodité pour rendre les fonctions de traitement d'image de
                  base \cite{imutils}
              \item[numpy] Permet d’effectuer des calculs numériques avec \gls{python} \cite{numpy}
              \item[opencv-python] Bibliothèque open-source pour la vision par ordinateur, l'apprentissage automatique
                  et le traitement des images \cite{opencv}
              \item[toml] Analyse et créer des fichierl TOML \cite{toml}
              \item[Mock.GPIO] Bibliothèque de \glspl{mock} pour la bibliothèque \gls{python} \textit{RPI.GPIO} \cite{mock_gpio}.
          \end{description}

    \item Nécessaire pour les tests
          \begin{description}
              \item[pytest] \Gls{framework} permettant d'écrire facilement des tests petits et lisibles \cite{pytest}
              \item[pytest-cov] Produit un rapport de couverture avec de tests réalisés avec \textit{pytest} \cite{pytest_cov}
              \item[pytest-mock] Fournit un injecteur de \gls{mock} \cite{pytest_mock}
              \item[flake8-pytest-style]
                  Vérifie les problèmes de style communs ou les incohérences \cite{flake8_pytest_style}.
          \end{description}

    \item Nécessaire pour la génération du \gls{paquet} du \gls{sae}
          \begin{description}
              \item[build] Outil de construction simple qui n'effectue aucune gestion des dépendances. \cite{build}
          \end{description}
\end{itemize}


\section{Installation avec \glsentryname{pip}}


\newpage
\section{Branchements des composants au Raspberry}
Le \gls{RPiCard} doit être alimenté avec une alimentation secteur. Un switch permet d'éteindre et d'allumer la carte.

\includegraphics[width=\textwidth]{img/Rpi4_pin.png}

\subsection{Branchements du servomoteur}

Pour le logiciel fournit, le servomoteur \gls{SG90} a des branchements spécifiques.
\begin{itemize}
    \item Fil marron : GND
    \item Fil rouge  : 5V
    \item Fil orange : GPIO 12
\end{itemize}

\subsection{Branchements du panneau de LEDs}

En considérant que vous êtes dos au panneau :
\begin{itemize}
    \item Brancher la borne de droite du bornier à la broche GPIO 5
    \item La borne du milieu doit rester vide
    \item Brancher la borne de gauche du bornier au GND
\end{itemize}

\subsection{Branchements de la caméra}
La caméra \gls{RPiCamera} est fournie avec une nappe. Côté caméra : Ouvrer le connecteur d'accueil de nappe en tirant sur la languette. 
Positionner la nappe de façon à ce que le trait de couleur bleu soit au dos de la lentille et fermer le connecteur. 
Faire de même sur le port d'accueil caméra du \gls{RPiCard}. Positionner le trait bleu face au connecteur Ethernet.
